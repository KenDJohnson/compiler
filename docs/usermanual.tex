\documentclass[12pt]{scrreprt}
\newcommand{\tab}{\hspace*{2em}}
\usepackage[english]{babel}
\usepackage[utf8]{inputenc}

\author{Ken Johnson}
\title{Pascal Compiler User Manual}

\begin{document}
\maketitle

The Pascal Compiler, pascalc, is a packaged Java program. As such it should be
completely cross platform, as long as you have JRE 1.7+ installed. The program
can be run with the Java Virtual Machine from a terminal like so:\\
\texttt{java -jar pascalc.jar <file.pas> }\\
The first parameter should be the filename of the Pascal code you wish to compile.
The MIPS assembly code will be generated to the current directory. The generated
file will have the file extension .asm, and the name of the file will be the
program name, i.e. the string following program, and preceding ; on the first
line of the Pascal file. Optionally the third command line argument can be
-p, which will print the syntax tree, which is the internal representation of
the Pascal code.\par\noindent

If errors are encountered during the compilation process, an error message
will be printed, along with the line number it occurred on. The program
will then exit with an error code corresponding to the error encountered.
These error codes can be seen as follows: \\

2: TOKEN\_NOT\_AVAILABLE will occur when something is not matched at the end. For
example, if there is no end. for the main there will not be a token available.\\

2: TOKEN\_MISMATCH will occur when a certain token is expected, but there was another
token in its place.\\



4: AFTER\_PROGRAM handles the error in which the semicolon after program is not 
present. \\


5: PROGRAM\_NOT\_FOUND is the error thrown when the first line does not
contain program \verb|<|programname ;\verb|>|. \\


8: EXPECTED\_EOF error occurs when there is extra text after the end. for the
main of the program. \\


9: COMPOUND\_STMT\_SEMICOLON is thrown when there is a semicolon after the
last statement in a compound statement. \\


10: VARIABLE\_NOT\_DEC occurs when an attempt to use a variable that has not
been declared. \\

11: ASSIGN\_REAL\_TO\_INT error is thrown when a real number is attempted to
be assigned into an integer variable. \\


12: REAL\_INT\_COMPARISON is thrown when a real number and an integer are compared. \\


\end{document}

\documentclass[12pt]{scrreprt}
\newcommand{\tab}{\hspace*{2em}}
\usepackage[english]{babel}
\usepackage[utf8]{inputenc}
\begin{document}
The testing for the recognizer consisted of looping through an array of
filenames and creating a new recognizer for each one, testing the
contents of the file to see if it is syntaxically correct. The testing
of the scanner is embedded in to this testing, as it must properly
recognize and the tokens and send them to the recognizer. The files
used are as follows:\\
\\
An empty Pascal file:\\
\\
\tab  program testProg;\\
\tab  begin\\
\tab	end\\
\tab .\\
\\
Testing declarations:\\
\\
\tab	program testDeclarations;\\
\\
\tab	var foo, bar : integer;\\
\tab	var foobar : real;\\
\tab	begin\\
\tab	end\\
\tab	.\\
\\
Testing array declarations:\\
\\
\tab	program testArray;\\
\tab	var listOfInts : array [ 47 : 22 ] of integer\\;
\tab	begin\\
\tab	end\\
\tab	.\\
	\\
Testing a function declaration with a compound declaration inside that
has one ID in it, should be a procedure call.\\
\\
\tab	program testArray;\\
\tab	var listOfInts : array [ 47 : 22 ] of integer;\\
\tab	function testFunc ( testParam : integer ) : integer ;\\
\tab	begin\\
\tab	random\\
\tab	end ;\\
\tab	begin\\
\tab	end\\
\tab	.\\
\\
Testing the same function declaration with two IDs in it, which should not
compile (and it does not)\\
\\
\tab	program testArray;\\
\tab	var listOfInts : array [ 47 : 22 ] of integer;\\
\tab	function testFunc ( testParam : integer ) : integer ;\\
\tab	begin\\
\tab	isnt correct\\
\tab	end ;\\
\tab	begin\\
\tab	end\\
\tab	.\\

\section{Second Phase, Parsing and Code Generation}

To ensure the compiler worked properly, a 	happy path'' Pascal file was
created. This tested edge cases for all different kinds of conditions.
In addition to this program that compiles properly, there are different 
programs to test that each of the errors will be handled properly.
\\
\subsection{Testing Good Code}
This first first program is a follows:\\


program happyPath ; \\
var foo, bar : integer ;\\
var foobar, pi : real ;\\
begin\\
\tab	foo := 2;\\
\tab	bar := 10;\\
\tab	pi := 3.1415 ;\\
\tab	foobar := pi / 2 ;\\
\tab	if foobar \verb|<| pi then\\
\tab\tab		write(foobar)\\
\tab	else\\
\tab\tab		begin\\
\tab\tab	end\\
\tab	if foobar \verb|>| pi then\\
\tab\tab		begin\\
\tab\tab 		end\\
\tab	else\\
\tab\tab		begin\\
\tab\tab\tab			write(pi)\\
\tab\tab		end\\
\tab	if 2 \verb|<|\verb|>| 3 then\\
\tab\tab		write(5)\\
\tab	else\\
\tab\tab		begin\\
\tab\tab		end\\
\tab	if 2 + 2 = 4 then\\
\tab\tab		write(4)\\
\tab	else\\
\tab\tab		begin\\
\tab\tab		end\\
\tab	if not (2 + 2 = 4) then\\
\tab\tab		write(4)\\
\tab	else\\
\tab\tab		begin\\
\tab\tab		write(5)\\
\tab\tab		end\\
\tab	foobar := 2*pi + 2 / 17 * 2.7183;\\
\tab	write(foobar);\\
\tab	while bar \verb|>| 0 do\\
\tab	begin\\
\tab\tab		bar := bar - 1;\\
\tab\tab		write(bar)\\
\tab	end\\
	\\

end . 



This starts by testing the declarations of integers and reals. Next basic assignment
is tested, followed by a series of if statements. Each of these if statements
check whether the boolean evaluations of expressions work properly
both for reals and integers, while simulaneously checking the write()
method. A more complicated assignment is created next, to ensure the proper
evaluation of a complex operation tree. Next a while loop is tested, to 
ensure it terminates at the proper step, and there are no off by one errors
in the code generation.


\subsection{Testing Bad Pascal Code}
The next series of tests consist of different cases that will cause compilation

TOKEN\_MISMATCH will occurr when a certain token is expected, but there was another
token in its place.

program tmm ; \\
var foo : integer ; \\
begin \\
\tab foo := < 23 \\
end . \\


AFTER\_PROGRAM handles the error in which the semicolon after program is not 
present.

program tmm  \\
var foo : integer ; \\
begin \\
\tab foo := < 23 \\
end . \\

PROGRAM\_NOT\_FOUND is the error thrown when the first line does not
contain program \verb|<|programname ;\verb|>|.

var foo : integer ; \\
begin \\
\tab foo := < 23 \\
end . \\

EXPECTED\_EOF error occurrs when there is extra text after the end. for the
main of the program.

program tmm ; \\
var foo : integer ; \\
begin \\
\tab foo :=  23 \\
end . \\
var foobar : integer ; \\


COMPOUND\_STMT\_SEMICOLON is thrown when there is a semicolon after the
last statement in a compound statement.

program tmm ; \\
var foo : integer ; \\
begin \\
\tab foo :=  23 ; \\
end . \\

VARIBLE\_NOT\_DEC occurrs when an attempt to use a variable that has not
been declared.

program tmm ; \\
var foo : integer ; \\
begin \\
\tab bar :=  23  \\
end . \\

ASSIGN\_REAL\_TO\_INT error is thrown when a real number is attempted to
be assigned into an integer variable.

program tmm ; \\
var foo : integer ; \\
begin \\
\tab foo :=  23.22  \\
end . \\

REAL\_INT\_COMPARISON is thrown when a real number and an integer are compared.

program tmm ; \\
var foo : integer ; \\
begin \\
if 1 \verb|>| 55.5 then \\
\tab	begin \\
\tab	end \\
else \\
\tab	begin \\
\tab	end \\
end . \\

\end{document}
